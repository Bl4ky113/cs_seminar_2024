\begin{frame}[allowframebreaks,allowdisplaybreaks]
    \section{B-Tree}
    \subsection{History}
    \frametitle{B-Tree History}
    \begin{columns}
        \begin{column}{0.5\textwidth}
            \begin{block}{}
                B-Trees where firstly studied, defined and implemented by R. Bayer and E. McCreight in 1972, using an IBM 360 series model 44 with an 2311 disk drive.
                \begin{figure}
                    \centering
                    \includegraphics[width=0.5\textwidth,height=\textheight,keepaspectratio]{resources/made/ibm360_44.png}
                    \caption[]{IBM 360 / 44}
                \end{figure}
            \end{block}
        \end{column}
        \begin{column}{0.5\textwidth}
            \begin{block}{}
                An IBM 360 series model 44 had from 32 to 256 \(KB\) of Random Access Memory, and weighed from 1,315 to 1,905 kg.
                \begin{figure}
                    \centering
                    \includegraphics[width=0.5\textwidth,height=\textheight,keepaspectratio]{resources/made/ibmdisk_2311.png}
                    \caption[]{IBM 2311 disk drive}
                \end{figure}
            \end{block}
        \end{column}
    \end{columns}

    \framebreak

    \begin{columns}
        \begin{column}{0.7\textwidth}
            \begin{block}{}
                \blockquote[Bayer and McCreight]{%
                    (\ldots) actual experiments show that it is possible 
                    to maintain an index of size 15.000 with an average of 9 retrievals, 
                    insertions, and deletions per second in real time on an IBM 360/44 
                    with a 2311 disc as backup store. (\ldots) it should be possible 
                    to main tain all index of size 1'500.000 with at least two transactions 
                    per second.}
            \end{block}
        \end{column}
        \begin{column}{0.3\textwidth}
            \begin{block}{}
                \begin{figure}
                    \includegraphics[height=0.3\textheight]{resources/made/r_bayer.png}
                    \caption[]{Rudolf Bayer}
                \end{figure}
                \begin{figure}
                    \includegraphics[height=0.35\textheight]{resources/made/McCreight.png}
                    \caption[]{Edward McCreight}
                \end{figure}
            \end{block}
        \end{column}
    \end{columns}
\end{frame}
