\begin{frame}[allowframebreaks,allowdisplaybreaks]
    \subsubsection{Code Implementation}
    \frametitle{AB-Tree differences with B-Trees---Code implementation}
    \begin{columns}
        \begin{column}{\textlecolumn}
            \begin{block}{}
                \begin{itemize}
                    \item In the implementation of a AB-Tree we will only replace the upper bound of the \emph{Branching factor}
                        by a \lstinline|BETA| constant. For example, in the AB-Tree structure.
                    \item It's important to define that this change of the upper bound won't change the rebalancing algorithms in a great way.
                \end{itemize}
                \inputminted[%
                    linenos,%
                    breakbytoken,%
                    highlightlines={1,2,6,7}%
                    ]{c}{resources/code/ab_tree_struct.c}
                \begin{itemize}
                    \item In the \lstinline|insert| operation, we can have some changes in some implementations.
                    \item Theese changes happen mainly on the \emph{Splitting}, of non-root pages, process in the rebalancing of the B-Tree.
                \end{itemize}
            \end{block}
        \end{column}
        \begin{column}{\textricolumn}
        \end{column}
    \end{columns}
    \framebreak{}
    \begin{columns}
        \begin{column}{\textlecolumn}
            \begin{block}{}
                \begin{itemize}
                    \item In the \emph{Splitting} process on a B-Tree, since we overflow the node, we will have \(2\alpha\) elements in 
                        the node, which we will split on the current node and a new node, resulting in two nodes with the minimun bound of \(\alpha\) elements.
                    \item But in AB-Trees, since for a node to overflow we could have the same or more elements, \(2\alpha\), in the overflowing node 
                        we have to decide which node will have to take the extra elements after each node gets the minimun elements.
                    \item In the current implementation, the balancing algorithm is just spliting the elements in half for each node, 
                        ending with two nodes with \(\frac{\beta}{2}\) elements.
                \end{itemize}
                \hspace{2cm}
                \begin{itemize}
                    \item In the current implementation there isn't any simple change to the \lstinline|delete| operation.
                    \item Since it depends mainly on the lower bound of the \emph{Branching factor}, \(\alpha\), which is the same for both types of tree.
                \end{itemize}
            \end{block}
        \end{column}
        \begin{column}{\textricolumn}
        \end{column}
    \end{columns}
\end{frame}
