\begin{frame}[allowframebreaks,allowdisplaybreaks]
    \section{AB-Tree}
    \subsection{AB-Tree Definition}
    \frametitle{AB-Tree Definition}
    \begin{columns}
        \begin{column}{\textlecolumn}
            \begin{block}{}
                \begin{itemize}
                    \item We will define that \(T\), an object, is a AB-Tree if they are an instance of the class.
                        \[
                            T \in \tau\left(\alpha, \beta, h\right)
                        \]
                    \item Where \(h\) is the height of the AB-Tree.
                    \item And, \(\alpha\) and \(\beta\) are predefined constants.
                    \item This is a tree based on B-Trees, which modifies the bounds of the B-Tree's \(\alpha\) constant.
                    \item The AB-Trees shares the height, keys and sub-trees properties with the B-Tree.
                    \item It mostly shares the operations with the B-Tree, such as 
                        \lstinline|find|, \lstinline|insert| and \lstinline|delete|, only having slight changes in some implementations.
                    \item With this we can say that, \textbf{every B-Tree is a AB-Tree but not every AB-Tree is a B-Tree}.
                    \item Also, an AB-Tree can be defined as
                        \[
                            T \text{ is a } \left(\alpha, \beta\right)\text{--Tree}
                        \] which is the most popular notation.
                    \item And we will keep using the same notation for a leaf, node and generic page from the B-Trees.
                \end{itemize}
            \end{block}
        \end{column}
        \begin{column}{\textricolumn}
        \end{column}
    \end{columns}
\end{frame}
