\begin{frame}[allowframebreaks,allowdisplaybreaks]
    \subsubsection{Delete}
    \frametitle{B-Tree Operations---Delete}
    \begin{columns}
        \begin{column}{\textlecolumn}
            \begin{block}{}
                \begin{itemize}
                    \item The deletion algorithm, just like the insert or find, in the B-Tree almost has nothing to share with any tree deletion algorithm.
                    \item Also, the first part is a \lstinline|find| algorithm where we are going to search if the key to delete exists and if it does
                        and it's position, and we store the nodes that we access and their pointer index on separated stacks.
                    \item Then, when reached a leaf with the value to delete, we just delete it. But now, we have to check 
                        for all the rebalancing cases.
                    \item If the current balancing node has a degree greater than \(\alpha\) we can stop the rebalancing process.
                    \item Then, if we are not on the root, we will check if our current node is not the last sub-tree on the parent node.
                    \item If the node isn't, we will check if the next neighbor node can share a key, or if it has more than \(\alpha\) keys.
                    \item In the case that the neighbor doesn't have \(\alpha\) elements we are going to join both nodes.
                    \item Then, we are going to check if the parent node needs some rebalancing and restart the rebalancing process.
                    \item Now, in the case that we are the the last sub-tree of the parent node we can't just chare elements with the next neighbor.
                    \item So we are just going to do the same thing but with the previous neighbor. Both process, the sharing or the join.
                    \item Also, if we reach the root on the rebalancing process, we check if the root has at least one key, and isn't a leaft at the same time.
                    \item But if the root doesn't have any element, we just return the root memory.
                    \item When we finally exit the rebalancing loop, we just return the object that we deleted.
                \end{itemize}
            \end{block}
        \end{column}
        \begin{column}{\textricolumn}
        \end{column}
    \end{columns}

    \inputminted[
        highlightlines={5, 6, 7, 11, 12, 13, 20, 21, 22, 25, 28, 34, 35, 36, 43, 44, 50, 52, 55, 69, 73, 77, 78, 79, 81, 82, 85, 88, 95, 97, 99, 101, 107, 108, 109, 111, 114, 117, 120, 122, 128, 129, 130, 132, 140, 143, 145, 146, 172, 173, 174, 177, 193, 194, 195, 197, 204}
    ]{c}{resources/code/b_tree_delete.c}
\end{frame}
